
\documentclass[12pt]{article}
%\usepackage[finnish]{babel}
\usepackage[T1]{fontenc}
\usepackage[utf8]{inputenc}
\usepackage{amssymb}
\usepackage{amsmath}
\usepackage{graphicx}
\usepackage{hyperref}
\newcommand{\pat}{\partial}
\newcommand{\be}{\begin{equation}}
\newcommand{\ee}{\end{equation}}
\newcommand{\bea}{\begin{eqnarray}}
\newcommand{\eea}{\end{eqnarray}}
\newcommand{\abf}{{\bf a}}
\newcommand{\Zcal}{{\cal Z}_{12}}
\newcommand{\zcal}{z_{12}}
\newcommand{\Acal}{{\cal A}}
\newcommand{\Fcal}{{\cal F}}
\newcommand{\Ucal}{{\cal U}}
\newcommand{\Vcal}{{\cal V}}
\newcommand{\Ocal}{{\cal O}}
\newcommand{\Rcal}{{\cal R}}
\newcommand{\Scal}{{\cal S}}
\newcommand{\Lcal}{{\cal L}}
\newcommand{\Hcal}{{\cal H}}
\newcommand{\hsf}{{\sf h}}
\newcommand{\half}{\frac{1}{2}}
\newcommand{\Xbar}{\bar{X}}
\newcommand{\xibar}{\bar{\xi }}
\newcommand{\barh}{\bar{h}}
\newcommand{\Ubar}{\bar{\cal U}}
\newcommand{\Vbar}{\bar{\cal V}}
\newcommand{\Fbar}{\bar{F}}
\newcommand{\zbar}{\bar{z}}
\newcommand{\wbar}{\bar{w}}
\newcommand{\zbarhat}{\hat{\bar{z}}}
\newcommand{\wbarhat}{\hat{\bar{w}}}
\newcommand{\wbartilde}{\tilde{\bar{w}}}
\newcommand{\barone}{\bar{1}}
\newcommand{\bartwo}{\bar{2}}
\newcommand{\nbyn}{N \times N}
\newcommand{\repres}{\leftrightarrow}
\newcommand{\Tr}{{\rm Tr}}
\newcommand{\tr}{{\rm tr}}
\newcommand{\ninfty}{N \rightarrow \infty}
\newcommand{\unitk}{{\bf 1}_k}
\newcommand{\unitm}{{\bf 1}}
\newcommand{\zerom}{{\bf 0}}
\newcommand{\unittwo}{{\bf 1}_2}
\newcommand{\holo}{{\cal U}}
\newcommand{\bra}{\langle}
\newcommand{\ket}{\rangle}
\newcommand{\muhat}{\hat{\mu}}
\newcommand{\nuhat}{\hat{\nu}}
\newcommand{\rhat}{\hat{r}}
\newcommand{\phat}{\hat{\phi}}
\newcommand{\that}{\hat{t}}
\newcommand{\shat}{\hat{s}}
\newcommand{\zhat}{\hat{z}}
\newcommand{\what}{\hat{w}}
\newcommand{\sgamma}{\sqrt{\gamma}}
\newcommand{\bfE}{{\bf E}}
\newcommand{\bfB}{{\bf B}}
\newcommand{\bfM}{{\bf M}}
\newcommand{\cl} {\cal l}
\newcommand{\ctilde}{\tilde{\chi}}
\newcommand{\ttilde}{\tilde{t}}
\newcommand{\ptilde}{\tilde{\phi}}
\newcommand{\utilde}{\tilde{u}}
\newcommand{\vtilde}{\tilde{v}}
\newcommand{\wtilde}{\tilde{w}}
\newcommand{\ztilde}{\tilde{z}}


\hoffset 0.5cm
\voffset -0.4cm
\evensidemargin -0.2in
\oddsidemargin -0.2in
\topmargin -0.2in
\textwidth 6.3in
\textheight 8.4in

\begin{document}

\normalsize

\baselineskip 14pt

\begin{center}
{\Large {\bf FYMM/MMP IIIa 2020 \ \ \  Problem Set 6}}
\end{center}

\bigskip

\noindent
Please submit your solutions for grading by \textbf{Monday 12.10.} in Moodle.


\begin{enumerate}
\item {\bf Mattress flipping.}
Bed mattress manufacturers recommend rotating a mattress twice a year. Let us consider 
the mathematics of mattress flipping. In Figure 1 are depicted the three ways of rotating a 
mattress by 180 degrees ({\em i.e.}, "flipping it") around the three symmetry axis, denoted
by {\bf R, P, Y}, plus the identity transformation {\bf I} (doing nothing). These operations form a group. 
(Assume that the mattress has patterns so that you can identify its upperside and underside, front and rear. Hint: you may use a sheet of paper.)
\begin{description}
\item [i)] Construct the Cayley (multiplication) table for the 4 operations {\bf I,R,P,Y}. 
\item [ii)] Can you identify the group of mattress flipping operations?
\end{description}
\begin{figure}[h]
\begin{center}
\includegraphics[scale=0.5]{mattress.PNG}
\caption{Mattress flipping.}
\end{center}
\end{figure}

\item Construct the character table of $\mathbb{Z}_2\times \mathbb{Z}_2$.

\item In the Example in the updated lecture notes, it was suggested that the number of conjugacy classes of a permutation group is equal to the number of distinct ``cycle types''.  Prove this statement for 
the permutation group $S_4$.  How many inequivalent irreducible unitary representations does $S_4$ have?

\item Given a vector space $V$, prove that every $\omega \in (V^*)^*$ can be uniquely associated with a vector $\vec{v}\in V$ such that $\omega (f)= \bra f , \vec{v}\ket$.

\item Let the (1,0)-tensor $R$
have the components
$$
R^1=a \ ; R^2 = a^2 \ ; R^3 = a^4 
$$
and the (0,1)-tensor $S$
have the components
$$
S_1=-b \ ; S_2 = c \ ; S_3 =- d \ . 
$$
Calculate all the components $T^\mu_\nu$ of the (1,1)-tensor $T=R\otimes S$.


\end{enumerate}
\end{document}
