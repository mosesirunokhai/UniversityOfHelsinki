
\documentclass[12pt]{article}
\usepackage[english,finnish]{babel}
\usepackage{t1enc}

% AMS packages:
\usepackage{amsbsy}
\usepackage{amsfonts}
\usepackage{amsmath}
\usepackage{amssymb}
\usepackage{amsthm}
\usepackage{amsxtra}
% for comments to work
\usepackage{verbatim}
\usepackage{hyperref}

%%%%%%%%%%%%%%%%%%%%%%%%%%%%%%%%%%%

% Hilbert spaces:
\newcommand{\hilb}{\mathcal{H}}
\newcommand{\banH}{\mathcal{B}(\hilb)}
\newcommand{\fock}{\mathcal{F}}

% Products:
\newcommand{\scalpr}[2]{( #1, #2 )}
\newcommand{\dualpr}[2]{\langle #1, #2 \rangle}

% Fonts
\newcommand{\calc}{\mathcal{C}}
\newcommand{\cala}{\mathcal{A}}
\newcommand{\calv}{\mathcal{V}}
\newcommand{\calf}{\mathcal{F}}
\newcommand{\cals}{\mathcal{S}}
\newcommand{\cald}{\mathcal{D}}
\newcommand{\banach}{\mathcal{B}}
\newcommand{\id}[1]{\mathbbm{1}\!\left({#1}\right)}

\newcommand{\ci}{{\rm i}}
\newcommand{\rmd}{{\rm d}}
\newcommand{\rme}{{\rm e}}

\newcommand{\re}{{\rm Re\,}}
\newcommand{\im}{{\rm Im\,}}

% misc
\newcommand{\defem}[1]{{\em #1\/}}
\newcommand{\vep}{\varepsilon}


\newcommand{\qand}{\quad\text{and}\quad}

% To define sets:
\newcommand{\defset}[2]{ \left\{ #1 \left|\, #2\makebox[0pt]{$\displaystyle\phantom{#1}$}\right.\!\right\} }

\newcounter{alplisti}
\renewcommand{\thealplisti}{\alph{alplisti}}
\newenvironment{alplist}[1][(\thealplisti)]{\begin{list}{{\rm #1}\ }{ %
      \usecounter{alplisti} %
    \setlength{\itemsep}{0pt}
    \setlength{\parsep}{0pt}  %
%    \setlength{\leftmargin}{5em} %
%    \setlength{\labelwidth}{5em} %
%    \setlength{\labelsep}{1em} %
%    \settowidth{\labelwidth}{(DR2)}
     \setlength{\topsep}{0pt} %
}}{\end{list}}

% Norms:
\newcommand{\abs}[1] {\lvert #1 \rvert}
\newcommand{\norm}[1]{\lVert #1 \rVert}
\newcommand{\floor}[1] {\lfloor {#1} \rfloor}
\newcommand{\ceil}[1]  {\lceil  {#1} \rceil}

% Basic spaces
\newcommand{\R} {\mathbb{R}}
\newcommand{\C} {{\mathbb{C}}}
\newcommand{\Rd} {{\mathbb{R}^{d}}}
\newcommand{\N} {\mathbb{N}}
\newcommand{\Z} {\mathbb{Z}}
\newcommand{\Q} {\mathbb{Q}}
\newcommand{\K} {\mathbb{K}}
\newcommand{\T} {\mathbb{T}}

%%%%%%%%%%%%%%%%%%%%%%%%%%%%%%%%%%%


\newcommand{\pat}{\partial}
\newcommand{\be}{\begin{equation}}
\newcommand{\ee}{\end{equation}}
\newcommand{\bea}{\begin{eqnarray}}
\newcommand{\eea}{\end{eqnarray}}
\newcommand{\abf}{{\bf a}}
\newcommand{\Zcal}{{\cal Z}_{12}}
\newcommand{\zcal}{z_{12}}
\newcommand{\Acal}{{\cal A}}
\newcommand{\Fcal}{{\cal F}}
\newcommand{\Ucal}{{\cal U}}
\newcommand{\Vcal}{{\cal V}}
\newcommand{\Ocal}{{\cal O}}
\newcommand{\Rcal}{{\cal R}}
\newcommand{\Scal}{{\cal S}}
\newcommand{\Lcal}{{\cal L}}
\newcommand{\Hcal}{{\cal H}}
\newcommand{\hsf}{{\sf h}}
\newcommand{\half}{\frac{1}{2}}
\newcommand{\Xbar}{\bar{X}}
\newcommand{\xibar}{\bar{\xi }}
\newcommand{\barh}{\bar{h}}
\newcommand{\Ubar}{\bar{\cal U}}
\newcommand{\Vbar}{\bar{\cal V}}
\newcommand{\Fbar}{\bar{F}}
\newcommand{\zbar}{\bar{z}}
\newcommand{\wbar}{\bar{w}}
\newcommand{\zbarhat}{\hat{\bar{z}}}
\newcommand{\wbarhat}{\hat{\bar{w}}}
\newcommand{\wbartilde}{\tilde{\bar{w}}}
\newcommand{\barone}{\bar{1}}
\newcommand{\bartwo}{\bar{2}}
\newcommand{\nbyn}{N \times N}
\newcommand{\repres}{\leftrightarrow}
\newcommand{\Tr}{{\rm Tr}}
\newcommand{\tr}{{\rm tr}}
\newcommand{\ninfty}{N \rightarrow \infty}
\newcommand{\unitk}{{\bf 1}_k}
\newcommand{\unitm}{{\bf 1}}
\newcommand{\zerom}{{\bf 0}}
\newcommand{\unittwo}{{\bf 1}_2}
\newcommand{\holo}{{\cal U}}
\newcommand{\bra}{\langle}
\newcommand{\ket}{\rangle}
\newcommand{\muhat}{\hat{\mu}}
\newcommand{\nuhat}{\hat{\nu}}
\newcommand{\rhat}{\hat{r}}
\newcommand{\phat}{\hat{\phi}}
\newcommand{\that}{\hat{t}}
\newcommand{\shat}{\hat{s}}
\newcommand{\zhat}{\hat{z}}
\newcommand{\what}{\hat{w}}
\newcommand{\sgamma}{\sqrt{\gamma}}
\newcommand{\bfE}{{\bf E}}
\newcommand{\bfB}{{\bf B}}
\newcommand{\bfM}{{\bf M}}
\newcommand{\cl} {\cal l}
\newcommand{\ctilde}{\tilde{\chi}}
\newcommand{\ttilde}{\tilde{t}}
\newcommand{\ptilde}{\tilde{\phi}}
\newcommand{\utilde}{\tilde{u}}
\newcommand{\vtilde}{\tilde{v}}
\newcommand{\wtilde}{\tilde{w}}
\newcommand{\ztilde}{\tilde{z}}

\selectlanguage{english}

\hoffset 0.5cm
\voffset -0.4cm
\evensidemargin -0.2in
\oddsidemargin -0.2in
\topmargin -0.2in
\textwidth 6.3in
\textheight 8.4in

\begin{document}

\normalsize

\baselineskip 14pt

\begin{center}
{\Large {\bf FYMM/MMP III \ \ \  Problem Set 1}}
\end{center}

\noindent
Please submit your solutions for grading by Monday 7.9.\ in Moodle (there is a link where you can do this after the exercise sheet).

\begin{enumerate}
\item Consider the following constructions; check each one whether it is a semigroup, monoid, group or
none of them. Why?
\begin{itemize}
\item The set of real numbers $\R$, with raising to power as multiplication:
$x\cdot y \equiv x^y$, $x,y\in \R$.
\item The set of positive natural numbers $\N_+=\{1,2,3,\ldots \}$ with the greatest common
divisor of $m,n\in \N_+$ as their product: $m\cdot n \equiv \text{gcd}(m,n)$.
\item The set of nonzero rational numbers $\Q\setminus\{0\}$, with the usual product
as multiplication: $(m/n) \cdot (p/q)
= (mp/nq)$.
\end{itemize}
\item Show that $|S_N|=N!$.
\item Consider the group $G=\{e,x_1,x_2,x_3,x_4,x_5\}$, where
\begin{eqnarray*}
e=\left( \begin{array}{ccc} 1 & 0 & 0 \\ 0 & 1 & 0 \\ 0 & 0 & 1
\end{array}\right) \ ; \
x_1 = \left( \begin{array}{ccc} 0 & 1 & 0 \\ 1 & 0 & 0 \\ 0 & 0 & 1
\end{array}\right) \\
x_2 = \left( \begin{array}{ccc} 0 & 0 & 1 \\ 0 & 1 & 0 \\ 1 & 0 & 0
\end{array}\right) \ ; \
x_3 = \left( \begin{array}{ccc} 1 & 0 & 0 \\ 0 & 0 & 1 \\ 0 & 1 & 0
\end{array}\right) \\
x_4 = \left( \begin{array}{ccc} 0 & 1 & 0 \\ 0 & 0 & 1 \\ 1 & 0 & 0
\end{array}\right) \ ; \
x_5 = \left( \begin{array}{ccc} 0 & 0 & 1 \\ 1 & 0 & 0 \\ 0 & 1 & 0
\end{array}\right) \ ,
\end{eqnarray*}
and the law of composition is the matrix multiplication.
Show that $G$ is isomorphic to a known group, give an explicit
construction of the isomorphism.
\item An equilateral triangle is symmetric under reflections, with
the line passing through the center and one of the vertices as the
reflection axis; and symmetric under 120 degree counterclockwise rotations
(with the center as the fixed point). Let $e$ be the identity
map (do nothing), $a$ a rotation by 120 degrees, and $b$ the above
mentioned reflection. Consider the group generated by
$e,a$ ja $b$ with composition of symmetry operations as the multiplication
rule.
What is the order of the group? (Hint: greater than three.)
Construct the multiplication table (Cayley table) of the group.
\end{enumerate}


\end{document}
