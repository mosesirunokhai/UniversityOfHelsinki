
\documentclass[12pt]{article}
\usepackage[english,finnish]{babel}
\usepackage{t1enc}

% AMS packages:
\usepackage{amsbsy}
\usepackage{amsfonts}
\usepackage{amsmath}
\usepackage{amssymb}
\usepackage{amsthm}
\usepackage{amsxtra}
% for comments to work
\usepackage{verbatim}
\usepackage{hyperref}

%%%%%%%%%%%%%%%%%%%%%%%%%%%%%%%%%%%

% Hilbert spaces:
\newcommand{\hilb}{\mathcal{H}}
\newcommand{\banH}{\mathcal{B}(\hilb)}
\newcommand{\fock}{\mathcal{F}}

% Products:
\newcommand{\scalpr}[2]{( #1, #2 )}
\newcommand{\dualpr}[2]{\langle #1, #2 \rangle}

% Fonts
\newcommand{\calc}{\mathcal{C}}
\newcommand{\cala}{\mathcal{A}}
\newcommand{\calv}{\mathcal{V}}
\newcommand{\calf}{\mathcal{F}}
\newcommand{\cals}{\mathcal{S}}
\newcommand{\cald}{\mathcal{D}}
\newcommand{\banach}{\mathcal{B}}
\newcommand{\id}[1]{\mathbbm{1}\!\left({#1}\right)}

\newcommand{\ci}{{\rm i}}
\newcommand{\rmd}{{\rm d}}
\newcommand{\rme}{{\rm e}}

\newcommand{\re}{{\rm Re\,}}
\newcommand{\im}{{\rm Im\,}}

% misc
\newcommand{\defem}[1]{{\em #1\/}}
\newcommand{\vep}{\varepsilon}


\newcommand{\qand}{\quad\text{and}\quad}

% To define sets:
\newcommand{\defset}[2]{ \left\{ #1 \left|\, #2\makebox[0pt]{$\displaystyle\phantom{#1}$}\right.\!\right\} }

\newcounter{alplisti}
\renewcommand{\thealplisti}{\alph{alplisti}}
\newenvironment{alplist}[1][(\thealplisti)]{\begin{list}{{\rm #1}\ }{ %
      \usecounter{alplisti} %
    \setlength{\itemsep}{0pt}
    \setlength{\parsep}{0pt}  %
     \setlength{\topsep}{0pt} %
}}{\end{list}}

% Norms:
\newcommand{\abs}[1] {\lvert #1 \rvert}
\newcommand{\norm}[1]{\lVert #1 \rVert}
\newcommand{\floor}[1] {\lfloor {#1} \rfloor}
\newcommand{\ceil}[1]  {\lceil  {#1} \rceil}

% Basic spaces
\newcommand{\R} {\mathbb{R}}
\newcommand{\C} {{\mathbb{C}}}
\newcommand{\Rd} {{\mathbb{R}^{d}}}
\newcommand{\N} {\mathbb{N}}
\newcommand{\Z} {\mathbb{Z}}
\newcommand{\Q} {\mathbb{Q}}
\newcommand{\K} {\mathbb{K}}
\newcommand{\T} {\mathbb{T}}

%%%%%%%%%%%%%%%%%%%%%%%%%%%%%%%%%%%


\newcommand{\pat}{\partial}
\newcommand{\be}{\begin{equation}}
\newcommand{\ee}{\end{equation}}
\newcommand{\bea}{\begin{eqnarray}}
\newcommand{\eea}{\end{eqnarray}}
\newcommand{\abf}{{\bf a}}
\newcommand{\Zcal}{{\cal Z}_{12}}
\newcommand{\zcal}{z_{12}}
\newcommand{\Acal}{{\cal A}}
\newcommand{\Fcal}{{\cal F}}
\newcommand{\Ucal}{{\cal U}}
\newcommand{\Vcal}{{\cal V}}
\newcommand{\Ocal}{{\cal O}}
\newcommand{\Rcal}{{\cal R}}
\newcommand{\Scal}{{\cal S}}
\newcommand{\Lcal}{{\cal L}}
\newcommand{\Hcal}{{\cal H}}
\newcommand{\hsf}{{\sf h}}
\newcommand{\half}{\frac{1}{2}}
\newcommand{\Xbar}{\bar{X}}
\newcommand{\xibar}{\bar{\xi }}
\newcommand{\barh}{\bar{h}}
\newcommand{\Ubar}{\bar{\cal U}}
\newcommand{\Vbar}{\bar{\cal V}}
\newcommand{\Fbar}{\bar{F}}
\newcommand{\zbar}{\bar{z}}
\newcommand{\wbar}{\bar{w}}
\newcommand{\zbarhat}{\hat{\bar{z}}}
\newcommand{\wbarhat}{\hat{\bar{w}}}
\newcommand{\wbartilde}{\tilde{\bar{w}}}
\newcommand{\barone}{\bar{1}}
\newcommand{\bartwo}{\bar{2}}
\newcommand{\nbyn}{N \times N}
\newcommand{\repres}{\leftrightarrow}
\newcommand{\Tr}{{\rm Tr}}
\newcommand{\tr}{{\rm tr}}
\newcommand{\ninfty}{N \rightarrow \infty}
\newcommand{\unitk}{{\bf 1}_k}
\newcommand{\unitm}{{\bf 1}}
\newcommand{\zerom}{{\bf 0}}
\newcommand{\unittwo}{{\bf 1}_2}
\newcommand{\holo}{{\cal U}}
\newcommand{\bra}{\langle}
\newcommand{\ket}{\rangle}
\newcommand{\muhat}{\hat{\mu}}
\newcommand{\nuhat}{\hat{\nu}}
\newcommand{\rhat}{\hat{r}}
\newcommand{\phat}{\hat{\phi}}
\newcommand{\that}{\hat{t}}
\newcommand{\shat}{\hat{s}}
\newcommand{\zhat}{\hat{z}}
\newcommand{\what}{\hat{w}}
\newcommand{\sgamma}{\sqrt{\gamma}}
\newcommand{\bfE}{{\bf E}}
\newcommand{\bfB}{{\bf B}}
\newcommand{\bfM}{{\bf M}}
\newcommand{\cl} {\cal l}
\newcommand{\ctilde}{\tilde{\chi}}
\newcommand{\ttilde}{\tilde{t}}
\newcommand{\ptilde}{\tilde{\phi}}
\newcommand{\utilde}{\tilde{u}}
\newcommand{\vtilde}{\tilde{v}}
\newcommand{\wtilde}{\tilde{w}}
\newcommand{\ztilde}{\tilde{z}}

\newcommand{\Mob}{\text{Mob}}


\selectlanguage{english}

\hoffset 0.5cm
\voffset -0.4cm
\evensidemargin -0.2in
\oddsidemargin -0.2in
\topmargin -0.2in
\textwidth 6.3in
\textheight 8.4in

\begin{document}

\normalsize

\baselineskip 14pt

\begin{center}
{\Large {\bf FYMM/MMP IIIa 2020 \ \ \  Problem Set 4}}
\end{center}

\bigskip

\noindent
Please submit your solutions for grading by \textbf{Monday 28.9.} in Moodle.

\begin{enumerate}
\item As discussed in supplementary notes, the presentation for the dihedral group $D_4$ is $D_4=\langle r, f| r^4, f^2, rfrf \rangle$, where
$r$ is a rotation by 90 degrees and $f$ is a reflection about a line through the midpoints of the edges. Generalize this for other dihedral groups
$D_n$ with $n\geq 5$. (In other words,  find and motivate the presentation for $D_n$).
\item Draw a picture of the braid (of 4 strands) $\sigma_3\sigma^{-1}_1\sigma^{-1}_2\sigma_3\sigma_1\sigma_3$.
\item Consider the left action of $SO(3)$ on the sphere $S^2\subset \mathbb{R}^3$ defined by the matrix-times-column-vector multiplication.
Parameterize the isotropy group of
$$
 x = \left( \begin{array}{c} -9/39 \\-60/65 \\ 4/13 \end{array} \right) \ .
$$ 

\item Consider the set of {\em M\"obius transformations}
\begin{equation}
\Mob  = \left\{f_A:\mathbb{C}\to\mathbb{C}|f_A(z) = \frac{az+b}{cz+d};\,A=\left(\begin{array}{cc} a&b\\ c&d  \end{array} \right)\in SL(2,\mathbb{C})  \right\}
\end{equation}
\begin{enumerate}
\item Show that $\Mob $ is a group, with composition of mappings as the product.
\item Show that the mapping
\begin{equation}
f: SL(2,\mathbb{C})\to \Mob ;\,f(A)=f_A
\end{equation}
is a homomorphism.
\item Find a subgroup $H$ of SL$(2,\mathbb{C})$  such that the quotient group SL$(2,\mathbb{C})$/$H$ is isomorphic to $\Mob $. Give reasons why.
\end{enumerate}
\item Let $V_1,V_2$ be vector spaces, $L: V_1\rightarrow V_2$ a linear map. Show that $Im L$ and $Ker L$ are vector subspaces of $V_1$ and $V_2$.
\end{enumerate}

\end{document}
%%%%%%%%%%%%%%%%%%%%%%%%%%
